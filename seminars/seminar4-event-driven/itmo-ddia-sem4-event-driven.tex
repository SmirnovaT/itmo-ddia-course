\documentclass{beamer}
\usetheme{Madrid}
\usepackage{minted}
% sudo tlmgr install minted
\usepackage[utf8]{inputenc}
\usepackage[T2A]{fontenc}
\usepackage[russian,english]{babel}
\usepackage{tikz}
\usetikzlibrary{arrows.meta, positioning}

% Footline: show current section (left) and frame numbers (right)
\setbeamertemplate{footline}{%
  \leavevmode%
  \hbox{%
    \begin{beamercolorbox}[wd=.78\paperwidth,ht=2.5ex,dp=1ex,left]{author in head/foot}%
      \hspace*{1em}\usebeamerfont{footline}\insertsectionhead%
    \end{beamercolorbox}%
    \begin{beamercolorbox}[wd=.22\paperwidth,ht=2.5ex,dp=1ex,right]{date in head/foot}%
      \usebeamerfont{footline}\insertframenumber{} / \inserttotalframenumber\hspace*{1em}%
    \end{beamercolorbox}%
  }%
}

\newcommand{\parpause}[1]{\only<+->{#1\par}}

\AtBeginSection[]
{
  \begin{frame}
    \frametitle{Содержание}
    \tableofcontents[currentsection]
  \end{frame}
}
\title{Семинар 4. Событийно-ориентированные архитектуры}
\subtitle{Принципы построения высоконагруженных систем}
\author{Георгий Семенов}
\institute{Институт прикладных компьютерных наук \\ Университет ИТМО}
\date{осень 2025}

\begin{document}

\frame{\titlepage}

\section{Введение}

\begin{frame}
  \frametitle{Оставшиеся активности на курсе}
  \begin{itemize}
    \item Защита домашнего задания 4 - сегодня после семинара, в понедельник и вторник
    \item Ликвидация долгов: до $\approx 20$ января, что нужно сделать – обсуждается индивидуально
  \end{itemize}
\end{frame}

\begin{frame}
  \frametitle{Вспоминаем лекцию I}
  \begin{itemize}
    \item Synchronous / Asynchronous
    \item Thread-Per-Request vs. Event Loop
    \item Очереди сообщений
    \begin{itemize}
        \item Гарантии доставки: at-most-once, at-least-once, exactly-once
        \item Dead Letter Queue (чтобы не останавливать конвейер)
        \item Transactional Outbox Pattern (чтобы делать композитные штуки без потери данных)
    \end{itemize}
    \item Kafka: быстро пишем в топик и читаем по указателю потребителями из партиций
    \item RabbitMQ: пишем в exchange, маршрутизируем по routing key, читаем из queue (\textsc{direct} = exact match, \textsc{fanout} = broadcast, \textsc{topic} = regex)
  \end{itemize}
\end{frame}

\begin{frame}
  \frametitle{Вспоминаем лекцию II}
  \begin{itemize}
    \item Теория массового обслуживания
    \begin{itemize}
        \item Нужен карман запаса для толерантности к bursts
        \item Формула Кингмана: среднее время ожидания в очереди заданий
    \end{itemize}
    \item Event-Driven Architecture
    \begin{itemize}
        \item Хореография – сервисы реагируют на события децентрализованно
        \item Оркестрация – сервисы реагируют в порядке обхода координатором
    \end{itemize}
  \end{itemize}
\end{frame}

\section{Event-Driven Architectures: Основы}

\begin{frame}
  \frametitle{EDA: основные понятия}
  \begin{itemize}
    \item Событие – иммутабельное сообщение, описывающее факт произошедшего действия
    \item Сервисы принимают/порождают сообщения в асинхронном режиме
    \begin{itemize}
        \item Передаем ID событий – event notification (обратный callback)
        \item Передаем событие целиком – event-carried state transfer (безопасность, нагрузка данными)
    \end{itemize}
  \end{itemize}
\end{frame}


\begin{frame}
  \frametitle{Форматы данных для представления событий}
  \begin{itemize}
    \item Сообщение – это объектная сущность (например, JSON, XML, Protobuf, Avro)
    \item Совместимость сообщений
    \begin{itemize}
        \item Backward - новые сервисы не ломаются от старых сообщений
        \item Forward - старые сервисы могут видеть новые поля
    \end{itemize}
    \item Schema Registry – централизованное хранилище схем сообщений
  \end{itemize}
\end{frame}

\begin{frame}[fragile]
  \frametitle{Эволюция схемы сообщения на примере protobuf}

    \textbf{Version 1:}
    \begin{minted}[escapeinside=||]{protobuf}
message Order {
    int32 order_id = 1;
    optional string customer_name = 2;
    optional double amount = 3;
    optional string status = 4;
}
    \end{minted}

\end{frame}

\begin{frame}[fragile]
  \frametitle{Эволюция схемы сообщения на примере protobuf}

    \textbf{Version 2:}
    \begin{minted}[escapeinside=||]{protobuf}
message Order {
    int32 order_id = 1;
    optional string customer_name = 2;
    optional double amount = 3 [deprecated=true];
    reserved 4;
    optional int64 timestamp = 5;
    optional string payment_method = 6;
}
    \end{minted}
    
\end{frame}

\section{Event Sourcing \& CQRS \& Sagas}

\begin{frame}
    \frametitle{Команда - поведенческий паттерн ООП}

    \begin{center}
        \includegraphics[width=0.8\linewidth,keepaspectratio]{images/oop_command.png}
    \end{center}
    
\end{frame}

\begin{frame}
    \frametitle{Event Sourcing: основные понятия}

    \begin{itemize}
        \item Источник истины – лог событий (Event Store), а не текущее состояние
        \item Каждое изменение состояния фиксируется как событие
        \item Состояние восстанавливается проигрыванием всех событий с начала
        \begin{itemize}
                \item Полный аудит и история всех изменений
                \item Временная машина – можем посмотреть состояние в любой момент
                \item Отладка: повторяем последовательность событий
        \end{itemize}
        \item Минусы: сложность, eventual consistency, размер хранилища (metadata overhead, retention, stabilization)
        \item Смотрим по ссылочке: \href{https://microservices.io/patterns/data/event-sourcing.html}{microservices.io}
    \end{itemize}

        
\end{frame}

\begin{frame}
    \frametitle{CQRS: Command Query Responsibility Segregation}

    \begin{itemize}
        \item Разделяем модели для чтения и записи
        \begin{itemize}
                \item Write model (Command) – оптимизирован для изменений (в микросервисах)
                \item Read model (Query) – оптимизирован для чтения (в специальном виде)
        \end{itemize}
        \item Event Sourcing + CQRS работают в паре
        \begin{itemize}
                \item События записываются в Event Store
                \item Из событий строится Read Model (денормализованное представление)
                \item Асинхронное обновление: eventual consistency
        \end{itemize}
        \item Смотрим по ссылочке: \href{https://microservices.io/patterns/data/cqrs.html}{microservices.io}
    \end{itemize}

        
\end{frame}

\begin{frame}
    \frametitle{SAGA: управление долгоживущими транзакциями}

    \begin{itemize}
        \item Долгоживущая транзакция – несколько микросервисов, которые нужно скоординировать
        \item Два подхода:
        \begin{itemize}
                \item Хореография (Choreography) – сервисы реагируют на события друг друга
                \item Оркестрация (Orchestration) – есть координатор (Saga Orchestrator)
        \end{itemize}
        \item Откат при ошибке: compensating transactions (выполняем обратные операции, aka undo-redo)
        \item Гарантия: каждый шаг либо успешен, либо откачен
        \item Посмотрим ссылочку здесь: \href{https://microservices.io/patterns/data/saga.html}{microservices.io}
    \end{itemize}

        
\end{frame}

\begin{frame}
    \frametitle{Лирическое отступление: ETL (Extract, Transform, Load)}

    \begin{itemize}
        \item Extract – извлекаем данные из источника
        \item Transform – преобразуем, очищаем, обогащаем
        \item Load – загружаем в хранилище
        \item Используется для интеграции разнородных систем
        \item Event-driven подход – streaming ETL (real-time вместо batch)
        \begin{itemize}
                \item Apache Kafka + Spark Streaming, Flink, Kafka Streams
                \item Меньше задержка, но сложнее в отладке
        \end{itemize}
    \end{itemize}

\end{frame}


\section{Serverless}

\begin{frame}
    \frametitle{Serverless: основные понятия}

    \begin{itemize}
        \item Не о том, что нет серверов, а о том, что нет управления серверами
        \item Function as a Service (FaaS) – платим за время выполнения
        \item Холодный старт (cold start) – первый запрос медленнее
        \item Идеален для event-driven архитектур
        \item Примеры: AWS Lambda, Google Cloud Functions, Azure Functions
        \item Минусы: vendor lock-in, сложность с состоянием, стоимость при высокой нагрузке
    \end{itemize}

        
\end{frame}

\begin{frame}
    \frametitle{Amazon Lambda}

    \begin{itemize}
        \item Платим за вычисления (GB-seconds): (память $\times$ время выполнения)
        \item Триггеры: S3, API Gateway, SQS, SNS, DynamoDB Streams, Kinesis, EventBridge
        \item Максимум 15 минут на выполнение (для долгих задач – асинхронные вызовы)
        \item Инструменты: SAM (Serverless Application Model), Serverless Framework
    \end{itemize}

\end{frame}


\section{Демо: сервис обработки изображений}

\begin{frame}
  \frametitle{Демо: сервис обработки изображений}
\begin{itemize}
    \item NB! Пример сгенерирован AI.
    \item Папка \textbf{queues} $\rightarrow$ \textbf{./start.sh}
\end{itemize}
    
\end{frame}

\section{Итоги}

\begin{frame}
  \frametitle{На кабана надейся, а сам...}
  
   \begin{center}
    \includegraphics[width=0.8\linewidth,keepaspectratio]{images/meme_ddia.png}
  \end{center}

\end{frame}

\begin{frame}
  \frametitle{Load balancing \& Work-life balancing}
  
   \begin{center}
    \includegraphics[width=0.6\linewidth,keepaspectratio]{images/meme_balancing.jpg}
  \end{center}

\end{frame}

\begin{frame}
  \frametitle{Bigtech нужен и для благих дел}
  
   \begin{center}
    \includegraphics[width=0.45\linewidth,keepaspectratio]{images/meme_bigtech.jpg}
  \end{center}

\end{frame}

\begin{frame}
  \frametitle{Спасибо!}

  Спасибо за терпение, работу и обратную связь!
  
  \begin{center}
    \includegraphics[width=0.45\linewidth,keepaspectratio]{images/thanks.jpg}
  \end{center}

\end{frame}

\begin{frame}
    \frametitle{Какие <<академические>> способы развития?}

    \begin{itemize}
        \item Книги:
        \begin{itemize}
            \item Release it! Проектирование и дизайн ПО для тех, кому не всё равно 
            \item Designing Data-Intensive Applications
            \item Google SRE Books
            \item Microservices Patterns
            \item Распределенные данные
            \item (*) Танненбаум. Операционные системы
            \item (*) Банда четырех. Паттерны проектирования 
        \end{itemize}
        \item Школа анализа данных Яндекса: трек <<Инфраструктура больших данных>>,
        поступление по альтернативному треку
        \item \href{https://www.youtube.com/@sukhoa}{Андрей Суховицкий} – лектор по проектированию и event sourcing
        \item Научные конференции (читать и обнаруживать направления): VLDB, SIGMOD, PODS, ICDT, Middleware, SOCC, EuroSys
    \end{itemize}

\end{frame}

\begin{frame}
  \frametitle{Итоги}
  \begin{itemize}
    \item Защищаем ДЗ-4, кому нужно – ликвидируем долги
    \item Отдыхаем и празднуем Новый год!
  \end{itemize}
\end{frame}

\end{document}