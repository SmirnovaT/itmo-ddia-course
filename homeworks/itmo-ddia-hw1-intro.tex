\documentclass[12pt]{article}
\usepackage[dvipsnames]{xcolor}
\usepackage{minted}
% sudo tlmgr install minted
\usepackage{hyperref}
\usepackage[margin=1in]{geometry} 
\usepackage{amsmath,amsthm,amssymb}
\usepackage{graphicx}
\usepackage[utf8]{inputenc}
\usepackage[T2A]{fontenc}
\usepackage[russian,english]{babel}

\hypersetup{
    colorlinks=true,
    linkcolor=blue,
    filecolor=magenta,      
    urlcolor=blue,
    pdfpagemode=FullScreen,
}

\newenvironment{problem}{
    \par\textbf{Задание: }\ignorespaces
}

\newenvironment{solution}
{
  \par\textbf{Решение:}\par\bigbreak
  \begingroup
}
{
  \par\bigbreak\hfill$\square$%
  \par\endgroup
}

\begin{document}
 
\title{
    \large Принципы построения высоконагруженных систем \\
    \normalsize Институт прикладных компьютерных наук ИТМО
    \bigbreak 
    \LARGE
    Домашнее задание 1. \\ Мониторинг и нагрузочное тестирование
    \bigbreak \normalsize
    Георгий Семенов \\
    \texttt{georgii.v.semenov@mail.ru} \\
    Мягкий дедлайн: Сб, 22.11.2025, 23:59 МСК \\
    Жесткий дедлайн: Сб, 29.11.2025, 23:59 МСК
}
\author{
    \color{red}{Имя Фамилия} \\
    \normalsize
    DDIA25-HW1-{\color{red}NameSurname}.pdf
}

\maketitle

\noindent\rule{\textwidth}{1pt} \bigbreak

{\large Правила курса}

\begin{itemize}
	\item Задание выдается на $2$ недели с двумя дедлайнами:
	      \begin{itemize}
	      	\item \textbf{Мягкий дедлайн} – в рамках этого периода можно заранее отправить решение
	      	      и получить обратную связь, чтобы исправить замечания до наступления жесткого дедлайна.
	      	\item \textbf{Жесткий дедлайн} – после этого новые посылки работы не принимаются, сдать работу больше нельзя.
	      \end{itemize}
	\item Планируется выдать $4$ домашних задания, каждое из которых стоит $\pm 12$ баллов. Правила выставления оценки за курс следующие:
	      \begin{itemize}
	      	\item \textbf{A -- <<5>>} – $\geq 90\%$ от общей суммы обязательных баллов (предв. $\geq 43.2$)
	      	\item \textbf{B/C -- <<4>>} – $\geq 75\%$ от общей суммы обязательных баллов (предв. $\geq 36$)
	      	\item \textbf{D/E -- <<3>>} – $\geq 60\%$ от общей суммы обязательных баллов (предв. $\geq 28.8$)
	      	\item \textbf{F -- <<2>>} – $< 60\%$ от общей суммы обязательных баллов (предв. $< 28.8$)
	      \end{itemize}
	\item Выполненное задание рекомендуется оформить в \LaTeX (например, в Overleaf) и выслать файлом с именем вида \texttt{DDIA25-HW1-IvanIvanov.pdf}
	      на почту выше.
	      \begin{itemize}
	      	\item Пожалуйста, оформляйте ответы на задания в блоке \texttt{Решение}.
	      \end{itemize}
\end{itemize}

\begin{tabular}{ |p{2cm}||p{1cm}|p{1cm}|p{1cm}|p{1.5cm}|p{1cm}|p{1cm}||p{1cm}||p{1cm}|  }
	\hline
	Задание & 1.1   & 1.2   & 1.3 & 1.4 (*) & 2.1   & 2.2   & $\Sigma$ & $\Sigma$ (*) \\
	\hline
	Баллы     & $1.5$ & $2.5$ & $3$ & $2$     & $1.5$ & $3.5$ & $12$     & $14$         \\
	\hline
\end{tabular}

\break

\section{Мониторинг}{}

\begin{flushleft}
	В ходе выполнения заданий в этом разделе вам предстоит помочь корпорации \textit{Монокль}
	внедрить актуальные стандарты мониторинга на основе Prometheus.
\end{flushleft}

\subsection{Метрики и их классы}

Ранее в \textit{Монокле} у каждого бизнес-юнита были свои политики мониторинга и надежности,
наиболее подходящие его продуктам и сервисам. Настал момент перейти к централизованной системе мониторинга
и внедрить единые стандарты телеметрии, в т.ч. сбора метрик.

\begin{problem}
(1.5 б.) Помогите SRE-команде \textit{Монокля} сформировать глоссарий терминов для нового корпоративного стандарта \textit{observability}.
Заполните таблицы ниже.
\end{problem}

\begin{solution}
	
	\begin{tabular}{ |p{4cm}|p{11cm}|  } \hline
		\textit{\href{https://www.ibm.com/think/insights/observability-pillars}{<<Observability Pillar>>}} & Характеристика               \\ \hline
		                                                                                                   &                              \\ \hline
		                                                                                                   &                              \\ \hline
		                                                                                                   &                              \\ \hline
	\end{tabular}
	
	\bigbreak
	
	\begin{tabular}{ |p{4cm}|p{8cm}|p{2cm}|  } \hline
		\textit{\href{https://sre.google/sre-book/monitoring-distributed-systems/}{<<Golden Signal>>}} & Пример метрики & Единица измерения    \\ \hline
		                                                                                               &                             &                                   \\ \hline
		                                                                                               &                             &                                   \\ \hline
		                                                                                               &                             &                                   \\ \hline
		                                                                                               &                             &                                   \\ \hline
	\end{tabular}
	
	\bigbreak
	
	\begin{tabular}{ |p{4cm}|p{11cm}|  } \hline
		\href{https://en.wikipedia.org/wiki/High_availability}{Метрика доступности} & Описание \\ \hline
		Uptime (\%)                                                                                   &                  \\ \hline
		Downtime (s)                                                                                  &                  \\ \hline
		MTBF (s)                                                                                      &                  \\ \hline
		MTTR (s)                                                                                      &                  \\ \hline
		MTTD (s)                                                                                      &                  \\ \hline
	\end{tabular}
	
\end{solution}

\break
\subsection{SLA}

Бизнес-юнит \textit{Монолит} сейчас использует собственную систему мониторинга \textit{Монада}
со своим форматом хранения метрик. Метрики \textit{Монады} хранятся в иерархической структуре
(например, \texttt{//monolith/web/http\_requests/timings}) и соответствуют лишь двум типам:
RPS-счётчики (\texttt{rate}) и перцентили (\texttt{percentile}). Поддерживаются только перцентили
из множества $(0.5, 0.75, 0.9, 0.95, 0.99)$.

\begin{flushleft}
	Настало время <<распилить \textit{Монолит}>> и канонично присоединиться к прогрессивному миру PromQL.
	SRE-команда \textit{БЮ Монолит} добродушно реализовала PromQL-адаптер к своей системе мониторинга.
	С ним можно выбрать метрику \textit{Монады} с помощью метки \texttt{monad}, указав путь \texttt{path},
	облачную зону \texttt{cloud\_dc} (поддерживаются зоны \texttt{AB, BC, CA}), тип поля и метрики
	\texttt{field} и \texttt{type} (что бы эта идиома ни значила).
\end{flushleft}

\begin{flushleft}
	Стажер в SRE-команде \textit{Монокля} получил задачу реализовать централизованный мониторинг SLO
	на основе предоставленного PromQL-адаптера для бэкенда \textit{Монолита} – 
	необходимо отслеживать 99-ый перцентиль времени ответа и суммарный RPS запросов:
\end{flushleft}

\begin{minted}{promql}
	avg (monad{
		cloud_dc=~".*",
		path="//monolith/web/http_requests/timings",
		field="p99",
		type="quantile"
	})
	
	sum (monad{
		cloud_dc=~".*",
		path="//monolith/web/http_requests/requests",
		field="1",
		type="rate"
	})
\end{minted}

\begin{problem}
Помогите SRE-команде \textit{Монокля} оценить и обосновать корректность результата их стажера.

\begin{enumerate}
	\item (0.75 б.) Приведите пример инцидента, при котором метрики стажера не зафиксируют реальную деградацию доступности сервиса для $>1\%$ пользователей \textit{Монолита}.
	\item (0.75 б.) Поможет ли использование взвешенного среднего вместо простого усреднения по зонам избежать проблемы в пункте выше? Проиллюстрируйте ответ.
	\item (0.5 б.) Предложите, как можно усовершенствовать предоставленный PromQL-адаптер, чтобы
	      помочь стажеру правильно составить необходимые запросы.
	\item (0.5 б.) Запишите корректные PromQL-выражения для 99-го перцентиля времени ответа и суммарного RPS запросов в предположении, что
	      эти метрики изначально были доступны как <<настоящие>> Prometheus-метрики, а не как адаптер через
	      \textit{label} с именем \texttt{monad}.
\end{enumerate}

\end{problem}

\begin{solution}
	
\end{solution}

\break
\subsection{Светофорные дашборды}

Разобравшись с деталями интеграции бизнес-юнитов корпорации, SRE-команда \textit{Монокля}
принялась за создание единой 24/7 дежурной смены и дашбордов для мониторинга всех сервисов корпорации.
Чтобы дежурный мог быстро оценить состояние систем, главный дашборд в Grafana должен быть оформлен
как множество \textit{светофорных плиток} – каждая плитка Stat отображает состояние одного сервиса:

\begin{itemize}
	\item Зеленая плитка \colorbox{green}{OK} означает, что все SLO сервиса выполняются.
	\item Желтая плитка \colorbox{yellow}{WARN} означает, что какой-то SLO сервиса находится в состоянии предупреждения.
	\item Красная плитка \colorbox{red}{CRIT} означает, что какой-то SLO сервиса не выполнен.
	\item Серая плитка \colorbox{gray}{?} означает, что какой-то из SLO не получилось посчитать (например, из-за No Data или ошибки запроса).
\end{itemize}

SRE-команда сразу же решила не заниматься накликиванием дашбордов вручную, а автоматизировать процесс
создания дашбордов на основе шаблонов и API Grafana. Для этого они хотят написать службу, которая конвертирует
описание SLO сервисов в красивый светофорный дашборд. Перед этим они разработали некое формальное исчисление
для описания SLO сервисов корпорации, в него вошли следующие конструкции:
\begin{itemize}
	\item $S_i :: service $ – сервис с номером $i$.
	\item $const :: scalar $ – можно объявить любую константу.
	\item $\{\text{UNK}, \text{CRIT}, \text{WARN}, \text{OK}\} :: state $ – монотонно возрастающее перечисление состояний многозначного SLO.
	\item $T_{99\%}(S_i) :: scalar? $ – время ответа (ms) в текущий момент у сервиса $S_i$ для $99\%$ пользователей.
	\item $R(S_i) :: scalar? $ – рейт запросов (rps) в текущий момент у сервиса $S_i$.
	\item $ < :: scalar? \rightarrow scalar? \rightarrow state $ – оператор сравнения скалярных величин; если одна из величин неизвестна, возвращает \text{UNK}; если условие верно, возвращает \text{OK}; если условие неверно, возвращает \text{CRIT}.
	\item $ \vee  :: state \rightarrow state \rightarrow state $ – оператор объединения условий, возвращающий наилучшее из состояний; если одно из состояний UNK, то возвращает UNK.
	\item $ \wedge  :: state \rightarrow state \rightarrow state $ – оператор объединения условий, возвращающий наихудшее из состояний; если одно из состояний UNK, то возвращает UNK.
\end{itemize}

Так, корректным термом этого исчисления считается выражение типа $state$, например:

$$(T_{99\%}(S_1) < 200) \wedge (2000 < R(S_1)) \wedge (R(S_1) < 10000) :: state $$

$$(20 < 300) \vee \text{WARN} = \text{OK} :: state $$


\begin{problem}
Помогите SRE-команде \textit{Монокля} реализовать транслятор их формализма в плитки Stat для дашборда в Grafana.

\begin{enumerate}
	\item (0.25 б.) Запишите терм для следующего SLO: <<$99\%$-время ответа сервиса $S_1$ должно быть больше $200$ мс для \text{WARN} и больше $300$ мс для \text{CRIT}>>.
	\item (0.5 б.) Отобразите в PromQL типы $\text{scalar}$ и $\text{state}$ и опишите, с помощью каких свойств Stat в Grafana необходимо раскрасить плитки в зависимости от значения $\text{state}$.
	\item (0.75 б.) Покажите, как реализовать операторы $<, \vee, \wedge$ предложенного исчисления в \href{https://prometheus.io/docs/prometheus/latest/querying/functions/}{PromQL} (подсказка: используйте выражения вида \texttt{OR on() vector(0)}).
	\item (1.5 б.) Предложите три SLO для параметров вашего ноутбука (docker-окружения), которые можно замерить
	      с помощью \texttt{node-exporter}. В окружении Grafana с семинара создайте дашборд из трех светофорных плиток (необязательно Stat, но с раскраской) для них.
	      Прикрепите три терма для ваших SLO, соответствующие им PromQL-выражения и скриншоты дашборда. Один из SLO должен содержать \texttt{WARN} уровень.
\end{enumerate}

\end{problem}

\begin{solution}
	
\end{solution}

\break
\subsection{Recording rules (*)}

SRE-команда \textit{Монокля} доказала свою компетентность в светофорных дашбордах, поразила своими
скиллами все продуктовые команды и уже почти обрадовала топ-менеджмент. Но на демонстрации произошла неловкость.

\begin{flushleft}
	На светофорной борде с интервалом $d = 5s$  было отображено более $n = 200$ светофорчиков, каждый из которых
	был реализован сложным PromQL-запросом с множеством операторов, рассмотренных выше. Prometheus не справлялся с этой нагрузкой,
	а Grafana подвисала при попытке отобразить дашборд. Топ-менеджмент был расстроен.
\end{flushleft}

\begin{flushleft}
	SRE-команда \textit{Монокля} решила исправить ситуацию с помощью \href{https://prometheus.io/docs/prometheus/latest/configuration/recording_rules/#recording-rules}{\textit{recording rules}}.
\end{flushleft}

\begin{problem}
\footnote{Задания и пункты, помеченные звездочкой, не являются обязательными.
	Эти баллы учитываются в общей сумме за курс, но являются дополнительными. Иными словами,
вы можете сделать дополнительное задание вместо какого-то обязательного и получить тот же суммарный балл.}
\begin{itemize}
	\item (0.5 б.) Объясните, как использование \textit{recording rules} поможет SRE-команде \textit{Монокля} в данной ситуации.
	\item (1 б.) Напишите \textit{rule group} для ваших метрик светофорных плиток из предыдущего задания в \texttt{prometheus.yml}.
	      Проверьте корректность вашего файла с помощью \texttt{promtool} и продемонстрируйте результат. Оцените, насколько это субъективно ускорило загрузку вашего дашборда при малом интервале обновления.
	\item (0.5 б.) Какие ограничения/недостатки у использования \textit{recording rules} вы можете назвать?
\end{itemize}
\end{problem}

\begin{solution}
	
\end{solution}

\break
\section{Нагрузочное тестирование}{}

\subsection{Закон Амдала}

Предположим, что программа выполняет операцию широковещательной рассылки (broadcast).  
Эта операция вносит дополнительное время выполнения, зависящее от числа задействованных ядер~$n$.  
Доступны две реализации broadcast:

\begin{itemize}
	\item первая добавляет параллельные накладные расходы $\beta = 0.0001n$;
	\item вторая --- $\beta = 0.0005 \cdot \ln(n)$.
\end{itemize}

\bigbreak

\begin{problem}
(1.5 б.) Для какого числа ядер достигается наибольшее ускорение программы с долей последовательного
кода$\alpha=0.001$ в каждой из реализаций?
\end{problem}

\begin{solution}
	
\end{solution} 

\break
\subsection{Case study}

В рамках этого задания вам предлагается \href{https://bytebytego.com/guides/real-world-case-studies/}{выбрать систему},
рассмотреть ее архитектуру, предложить оценку нагрузки и SLA, выявить \textit{bottlenecks} (см. лекцию) и
предложить метрики для мониторинга (см. задание 1.1). 

\begin{problem}
\begin{enumerate}
	\item (1 б.) Опишите архитектуру и компоненты выбранной вами системы. Оцените внешнюю входную нагрузку и гарантии: throughput, latency, количество пользователей, объем хранимых данных.
	      Прокомментируйте ваши оценки.
	\item (1.25 б.) Выявите 3-5 потенциальных \textit{bottlenecks} – \href{https://dictionary.cambridge.org/dictionary/english/speculate}{проспекулируйте}, в каких частях системы они могут возникнуть? Рассмотрите отдельно \textit{database bottlenecks}.
	\item (1.25 б.) Предложите метрики и их SLO для покрытия предложенных вами \textit{bottlenecks}. Какие стратегии автоматического реагирования можно предложить в случае срабатывания алертов на них?
\end{enumerate}
\end{problem}

\begin{solution}
	
\end{solution} 
 
\end{document}